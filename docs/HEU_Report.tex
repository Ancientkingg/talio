\documentclass[sigconf]{acmart}

% meta-data
\title{OOP Project Report -- Group 73}
\author{Samuel Bruin, Rafayel Gardishyan, Tejas Kochar, Jannes Kelso, Danylo Kozak}

% remove overhead that is used in regular ACM papers
\settopmatter{printacmref=false} % box after abstract
\renewcommand\footnotetextcopyrightpermission[1]{} % copyright on first page
\pagestyle{plain} % running headers

\usepackage[utf8]{inputenc}
\usepackage[T1]{fontenc}

\begin{document}

\begin{abstract}
% !TEX root =  ../report.tex
A clear and well-documented \LaTeX\ document is presented as an
article formatted for publication by ACM in a conference proceedings
or journal publication. Based on the ``acmart'' document class, this
article presents and explains many of the common variations, as well
as many of the formatting elements an author may use in the
preparation of the documentation of their work.

\end{abstract}

\maketitle

% !TEX root =  ../report.tex
\section{Introduction}\label{sec:introduction}

During the development process of any application, it is important to carefully consider the user experience and all the possible interactions a given user could have with your project.
Doing so allows for better understanding of the target audience, as well as providing a comprehensive overview of the strengths and flaws of the current iteration of the project.
The Heuristic Usability Evaluation (HUE) report is centred around identifying key areas of improvement within a prototype, with the ultimate goal being to improve the overall usability, appearance and functionality of said prototype in a precise manner.
This is done with the help of a group of experts, which provide an external, non-biased review of the application.
In the context of the given prototype, the HUE will be focused on the user-friendliness of the UI, with an emphasis on key UI pylons such as ``User Control and Freedom'' and ``Recognition rather than Recall''.
The combined feedback for all the covered criteria will serve as the basis for a more comprehensive, unified evaluation.
The prototype being evaluated is a (semi) functional model of talio.
- a sleek and minimal task management application.
The prototype, created with moqups, provides an interactive overview of the main features and scenes featured in the application.
The prototype features minimalistic style language, a black/grey/white color scheme and a simple, non-cluttered UI which presents all the functionality to the user immediately.
It can be found at the following link: \url{https://app.moqups.com/XrwUxfXBNqZIBTLGVkUx5VFnTulW9qBk/view/page/ad64222d5}

 (Guys im not sure how to format this stuff so I'll just put the report here for now)

    During the development process of any application, it is important to carefully consider the user experience and all of the possible interactions a given user could have with your project. Doing so allows for better a understanding of the target audience, as well as providing a comprehensive overview of the stregnths and flaws of the current iteration of the project. The Heuristic Usability Evaluation (HUE) report is centered around identifying key areas of improvement within a prototype, with the ultimate goal being to improve the overall usability, appearance and functionality of said prototype in a precise manner. This is done with the help of a group of experts, which provide an external, non biased review of the application. In the context of the given prototype, the HUE will be focused on the user friendliness of the UI, with an emphasis on key UI pylons such as "User Control and Freedom" and "Recognition rather than Recall". The combined feedback for all of the covered criteria will serve as the basis for a more comprehensive, unified evaluation. 
    The prototype being evaluated is a (semi) functional model of talio. - a sleek and minimal task management application. The prorotype, created with moqups, provides an interactive overview of the main features and scenes featured in the application. The prototype features minimalistic style language, a black/grey/white color scheme and a simple, non-cluttered UI which presents all of the functionality to the user immediately. It can be found at the following link: https://app.moqups.com/XrwUxfXBNqZIBTLGVkUx5VFnTulW9qBk/view/page/ad64222d5 (I'm not sure what they mean with "show the prototype" so this will do for now).

Methods

    In order to properly conduct the Heuristic Usability Evaluaiton, 5 (I think 5 members in their group right?) experts were recruited to offer their reviews on the prototype. The experts currently follow a BSc in Computer Science and Engineering, and are all accustomed to modern technology standards, making them suitably qualified to conduct a heuristic report of the prototype. Their prior experience and interest in Software Development and general technology makes them extremely sharp with regards to identifying and isolating flaws within an application's UI. 

    In order to conduct the HUE, the experts were given a standard procedure to follow, which allowed them to get an overview of the entire application and all of its moving parts. The procedure was relayed to the experts in the form of a manual on Google Docs, with a set of instructions and criteria to focus on during the evaluation. The prototype presented to the experts for review was an interactive and navigable mock up of the final application, which allowed the experts to get a feel of how a user might actually use the program.
    First, the experts were instructed to go over the entire prototype once. Since the prototype was functional, this could be done with the use of built in navigation tools, providing a fully immersive experience to the reviewers. After doing a preliminary run through the applicatio, the experts were asked to list any general flaws they faced in the UI. They were asked to use a 4 step format, consisting of: 1. Problem Description, 2. Likely/actual Difficulties, 3. Specific Contexts, 4. Assumed Causes, which allowed the experts to identify issues as well as pinpoint the contexts in which these issues were most likely to appear. 
    Next, the experts were asked to identify errors specifically related to each of the 8 curated Heuristics criteria listed in the google docs. The criteria correspond to the industry standard Heruistic model, involving the following 8 points: 1. Visibility of system status, 2. Match between system and the real world, 3. User control and freedom, 4. Consistency and standards, 5. 
Recognition rather than recall, 6. Flexibility and efficiency of use, 7. Aesthetic and minimalist design, 8. Help and documentation. The aformentioned criteria were briefly described in the document, with much more documentation available online. For each specific criteria, the experts were encouraged to provide criticism in the following format: Page; Type; Category; Description, which greatly helps in identifying exactly where and how any issues are occuring, making them easier to address in the development process. 
    Heuristics is a largely qualitative area and should be adressed as such. Given the aformentioned criteria and structure, the experts were able to provide qualitiative feedback which could be categorized by where it occured, and the Heuristics category. While numerical data may be easier to visualize and model, the qualitative feedback provided serves as a perfect foundation to specific issues that can be adressed in the development process, and specifically, placed on gitLab to be resolved by a discrete Merge Request.






% !TEX root =  ../report.tex
\section{Citations}

You can cite papers, e.g., \cite{test-ref}.
To make the references appear, make sure to compile the latex sources, then bibtex, and then latex twice.

\bibliographystyle{ACM-Reference-Format}
\bibliography{references}

\end{document}
\endinput
