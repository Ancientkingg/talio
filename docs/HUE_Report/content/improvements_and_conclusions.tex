% !TEX root =  ../report.tex
\section{Improvements and Conclusions}


\vspace{2mm}

In the initial join server page, we will:
\begin{itemize}
    \item briefly describe what is expected from the user, so that non-technical users are not confused
    \item provide a default server address for users to connect to
    \item provide users with the option of hosting their own server
    \item guide users to provide a valid server address with precise and helpful error messages. The cases we have decided on are for when a user does not provide any address or provides an invalid address. In case the server is not responding, this will also be conveyed in a non-technical, user-friendly manner.
\end{itemize}

\vspace{2mm}

To the home page, we will:
\begin{itemize}
    \item provide a way for users to disconnect from the current server so that they can join another server
    \item make it more evident that this page is the dashboard, perhaps by putting such a label at a prominent position near the top of the page
    \item more clearly indicate that the board settings can be accessed by clicking on the board title
    \item clearly indicate how to copy the board invite code by changing the paperclip icon to text reading "Copy Code" and moving the button into the upper bar aside the settings and tag buttons
    \item the lists currently have a pen indicating an editing feature, but those will be replaced by X's to indicate deletions, as the only other feature, renaming, can be done by clicking on the title of the list
    \item the edit tags modal had not been implemented yet in the prototype, but will follow a similar minimal style as the rest of the app and have the following features:
    \begin{itemize}
        \item OK (finalize changes and leave modal)
        \item Cancel (revert changes and leave modal)
        \item Add new tag and set name
        \item Remove tag
        \item Rename tag
        \item Change tag color
    \end{itemize}
\end{itemize}

\vspace{2mm}

The Board overview was the page with the most features, and also the most reported problems. To fix these, we will
\begin{itemize}
    \item make the board settings easier and more intuitive to find, maybe by making the pencil icon beside it visible at all times instead of just showing it when a mouse hovers over the name of the board.
    \item move the copy board button to the bar at the top of the screen instead of being in the body. We will also make it more easy to identify.
    \item provide cards with both a description and a title, as opposed to just a block of text as they have now.
\end{itemize}