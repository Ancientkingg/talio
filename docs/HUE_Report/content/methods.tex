\section{Methods}

In order to properly conduct the Heuristic Usability Evaluation, 5 (I think 5 members in their group right?) experts were recruited to offer their reviews on the prototype.
The experts currently follow a BSc in Computer Science and Engineering, and are all accustomed to modern technology standards, making them suitably qualified to conduct a heuristic report of the prototype.
Their prior experience and interest in Software Development and general technology makes them extremely sharp in regard to identifying and isolating flaws within an application's UI\@.

In order to conduct the HUE, the experts were given a standard procedure to follow, which allowed them to get an overview of the entire application and all of its moving parts.
The procedure was relayed to the experts in the form of a manual on Google Docs, with a set of instructions and criteria to focus on during the evaluation.
The prototype presented to the experts for review was an interactive and navigable mock up of the final application, which allowed the experts to get a feel of how a user might actually use the program.
First, the experts were instructed to go over the entire prototype once.
Since the prototype was functional, this could be done with the use of built-in navigation tools, providing a fully immersive experience to the reviewers.
After doing a preliminary run through the application, the experts were asked to list any general flaws they faced in the UI. They were asked to use a 4-step format, consisting of:

\begin{enumerate}
    \item Problem Description
    \item Likely/actual Difficulties
    \item Specific Contexts
    \item Assumed Causes
\end{enumerate}

which allowed the experts to identify issues as well as pinpoint the contexts in which these issues were most likely to appear.
Next, the experts were asked to identify errors specifically related to each of the 8 curated Heuristics criteria listed in the Google Docs.
The criteria correspond to the industry standard Heuristic model, involving the following 8 points:
\begin{enumerate}
    \item Visibility of system status
    \item Match between system and the real world
    \item User control and freedom
    \item Consistency and standards,
    \item Recognition rather than recall
    \item Flexibility and efficiency of use
    \item Aesthetic and minimalist design
    \item Help and documentation.
\end{enumerate}
The aforementioned criteria were briefly described in the document, with much more documentation available online.
For each specific criteria, the experts were encouraged to provide criticism in the following format: Page; Type; Category; Description, which greatly helps in identifying exactly where and how any issues are occurring, making them easier to address in the development process.
Heuristics is a largely qualitative area and should be addressed as such.
Given the aforementioned criteria and structure, the experts were able to provide qualitative feedback which could be categorised by where it occurred, and the Heuristics category.
While numerical data may be easier to visualize and model, the qualitative feedback provided serves as a perfect foundation to specific issues that can be addressed in the development process, and specifically, placed on gitLab to be resolved by a discrete Merge Request.

