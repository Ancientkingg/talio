\section{Report}

This section contains the problems reported by evaluators, grouped by page and ordered by perceived severity of the issues.

\subsection{Initial Join Server}
The most severe issue reported for this page was that it was not very intuitive and lacked the necessary descriptions or instructions without which the new users could feel lost. Two evaluators also pointed out that a default server address was missing, which would make things much more evident. A default port number was also requested. \newline
Another piece of feedback, was that we need to rigorously error check and throw corresponding error messages on the "Server Connection" screen, so that users know what's going wrong.

\subsection{Home}

A very general issue raised by a reviewer was the concern that users may be disoriented when first arriving on the Home page if they haven't used the app before. A proposed way of fixing this is to add a "Workspace" or "Dashboard" label to the page. Changing the header of the page to read "Talio: Dashboard" could certainly help some users understand exactly what they are looking at and make sense of the various buttons and boards.

One piece of functionality that we forgot to display in the prototype is the ability to disconnect from the current server and reconnect to a different one. The current prototype would not allow for any such change: after connecting to a server once, it would not reopen the "Server Connection" screen as it does the first time that one launches the app. This was simply an oversight in prototype design and we will make sure to have an option for this on the home page, which reopens the "Server Connection" screen with an additional "Cancel" option, as planned.

\subsection{Board Overview}

An issue users may face on the Board Overview page is finding the Board settings. The prototype allows access to this by clicking on the board name at the top of the screen, which also displays a pencil icon, indicating editing, when hovering over it. A reviewers speculation as to why this may not feel so intuitive, was that the board name is very large and feels more like a static title than a functional element.

A similar issue was raised about how copying the board code feels a bit unintuitive. The code can be found both under the board settings modal as well as copied by clicking the paperclip icon at the top right.

A reviewer noted that (if we wish to complete the advanced features) cards should have both a title and a description. They will have these features, they were just not clearly communicated in the prototype.

One reviewer noted that the edit list and edit tags modals are not implemented in the prototype.

Another reviewer expressed concerns that some of the mock cards don't have tags, don't have titles, or don't have descriptions, and that this may confuse users. The prototype is not fully fleshed out and therefore not all the cards are entirely prepared. Only some of the cards had all the features implemented to briefly demonstrated what that will look like. In the final application the user will only be presented with a single empty list upon creating a board so there is no need to worry about the users impression of the specific mock cards in the prototype. They can add tags as they wish and cards will require a title.

\subsection{Client Settings}

All the reviewers were satisfied with the settings.
    
